\subsection{QUEST method}

The attitude determination is performed using the measurements coming from the Sun sensor and from the star tracker. The measurements of the magnetic field will be used just for control purposes. The chosen algorithm is the QUaternion ESTimation (\textit{QUEST}) method. It is a solution of the Wahba's problem which allows for the estimation of the quaternion describing the relative orientation of the spacecraft with respect to the inertial frame. When the Sun illuminates the Sun sensor, the algorithm processes its measurement along with three directions coming from the star tracker, for a total of $N = 4$ unit vectors. In this case, the set of the exact directions in the inertial frame and the set of measurements in the body frame will be:

\begin{equation}
    \{ \hat{\mathcal{V}}_n \} = \{ \hat{\mathbf{S}}_n, \, \hat{\mathbf{v}}_{n,1}, \, \hat{\mathbf{v}}_{n,2}, \, \hat{\mathbf{v}}_{n,3} \} \text{  and  } \{ \Tilde{\mathcal{V}}_b \} = \{ \Tilde{\mathbf{S}}_b, \, \Tilde{\mathbf{v}}_{b,1}, \, \Tilde{\mathbf{v}}_{b,2}, \, \Tilde{\mathbf{v}}_{b,3} \} 
\end{equation}

When the Sun sensor is not in sunlight, the algorithm relies on the tracking of four stars. The sets will then become:

\begin{equation}
    \{ \hat{\mathcal{V}}_n \} = \{ \hat{\mathbf{v}}_{n,1}, \, \hat{\mathbf{v}}_{n,2}, \, \hat{\mathbf{v}}_{n,3}, \, \hat{\mathbf{v}}_{n,4} \} \text{  and  } \{ \Tilde{\mathcal{V}}_b \} = \{ \Tilde{\mathbf{v}}_{b,1}, \, \Tilde{\mathbf{v}}_{b,2}, \, \Tilde{\mathbf{v}}_{b,3}, \, \Tilde{\mathbf{v}}_{b,4} \} 
\end{equation}

The algorithm consists in finding the eigenvector associated to the maximum eigenvalue of a 4x4 matrix $K$. From a statistical point of view, this eigenvector represents the optimal attitude of the spacecraft (in the form of a quaternion) as it minimizes the Wahba cost function \cite{bernelli}. $K$ is composed as following:

\begin{equation}
    K = \begin{bmatrix}
    S - \sigma \cdot \mathbb{I}_3 & Z \\
    Z^T & \sigma
    \end{bmatrix}
    \text{ with: } 
    \begin{cases}
    B = \sum_{i=1}^{N=4} \alpha_i \,  \{ \Tilde{\mathcal{V}} \}_i \, \{ \hat{\mathcal{V}} \}_i^T \\
    S = B + B^T \\
    Z = [B_{2,3} - B_{3,2}, \, B_{3,1} - B_{1,3}, \, B_{1,2} - B_{2,1}]^T \\
    \sigma = \mathrm{trace} \, [B]
    \end{cases}
\end{equation}

$\alpha_i$ are weights depending on the relative precision of the sensors, such that:

\begin{equation}
    \alpha_i = \frac{1/\sigma_i}{\sum 1/ \sigma_i}
\end{equation}

The terms $\sigma_i$ represent the accuracy of the sensors expressed in degrees: for the Sun sensor, it is reported in \cref{tab:sun_sensor}, while for the star tracker measurements it is the root sum of squares of twice the cross-boresight accuracy plus the roll-boresight accuracy ($\sigma_{ST} = \sqrt{\sigma_{cross}^2 + \sigma_{cross}^2 + \sigma_{roll}^2}$).

The problem is solved using the command \mcode{eig} in MATLAB, from which the estimation of the quaternion $\bar{q}$ is retrieved. Using \cref{eq:dcm}, also the estimated DCM $\bar{A}_{b/n}$ can be computed. It is also possible to estimate the angular velocity $\bm{\omega}_{\bar{q}}$ through the pseudo-inverse of a matrix $Q$ and from the numerical derivative of the quaternion $\dot{\bar{q}}$:

\begin{equation}
    \bm{\omega}_{\bar{q}} = 2\, Q^* \dot{\bar{q}} \text{ with  } Q = 
    \begin{bmatrix}
    \bar{q}_4 & - \bar{q}_3 & \bar{q}_2 \\
    \bar{q}_3 & \bar{q}_4 & -\bar{q}_1 \\
    -\bar{q}_2 & \bar{q}_1 & \bar{q}_4 \\
    -\bar{q}_1 & -\bar{q}_2 & -\bar{q}_3
    \end{bmatrix}
\end{equation}

The quaternion $\bar{q}$ is filtered before derivation. Even if the noise is wide-band, the components at high frequency (far from the frequencies concerning the rotational dynamics) can be removed using a low-pass filter. The cut-off frequency is set equal to $1\,rad/s$ after the analysis of the frequency spectrum and with trial-and-error strategy. The numerical derivation is implemented as a diagonal transfer function matrix in the Laplace domain, whose components are:

\begin{equation}
    D_i (s) = \frac{s}{1 + \tau_i s}
\end{equation}

The poles $1/\tau_i = 10 \; rad/s$ have been chosen in order to have unitary gain at frequencies beyond the band of interest \cite{dozio}.

\subsection{Extended State Observer (ESO)} \label{sec:ESO}

The Extended State Observer (\textit{ESO}) \cite{biggs} is implemented in order to further filter the estimation of the angular velocity $\bm{\omega}_{\bar{q}}$ and to estimate the unknown disturbances acting on the satellite. \cref{eq:ESO} describes the evolution of the estimated angular velocity $\bar{\bm{\omega}}$ and of the estimated disturbance torque $\bar{\mathbf{M}}_d$.

\begin{subequations}
    \begin{empheq}[left=\empheqlbrace]{align}
        & \dot{\bar{\bm{\omega}}} = I^{-1}\, (I \bar{\bm{\omega}} \times \bar{\bm{\omega}} + \bar{\mathbf{M}}_c + \bar{\mathbf{M}}_d) + L_{\omega}\,(\bm{\omega}_{\bar{q}} - \bar{\bm{\omega}}) \\
        & \dot{\bar{\mathbf{M}}}_d = L_d\, (\bm{\omega}_{\bar{q}} - \bar{\bm{\omega}})
    \end{empheq}
    \label{eq:ESO}
\end{subequations}

$\mathbf{M}_c$ is the estimated control torque, which can be computed once that the control logic is defined through measured and estimated quantities. The gains are set equal to $L_w = 0.01$ and $L_d = 5 \cdot 10^{-5}$ after trial-and-error iterations. The term $\bar{\mathbf{M}}_d$ will be used in the design of the control laws as it can increase the robustness and the disturbance rejection capabilities of the control system.